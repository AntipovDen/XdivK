\documentclass{article}

\usepackage{amsmath}
\usepackage{algorithmicx}
\usepackage{algpseudocode}
\usepackage{cite}
\usepackage{flushend}
\usepackage{graphicx}
\usepackage{graphics}
\usepackage{url}
\usepackage[table]{xcolor}
\usepackage{xspace}
\usepackage[T2A]{fontenc}

\usepackage{algorithm}
\usepackage[english]{babel}

\usepackage[utf8]{inputenc}

\newcommand{\OM}{\textsc{OneMax}\xspace}
 \newcommand{\J}{\textsc{Jump}\xspace}
\newcommand{\LB}{\textsc{LeftBridge}\xspace}
\newcommand{\RB}{\textsc{RightBridge}\xspace}
\newcommand{\EARL}{\textsc{EA+RL}\xspace}
\newcommand{\RLS}{\textsc{RLS}\xspace}
\newcommand{\OMZM}{\textsc{OneMax+ZeroMax}\xspace}
\newcommand{\XdK}{\textsc{XdivK}\xspace}

\allowdisplaybreaks

\begin{document}

\section{Problem statement}

We consider simple $(1 + 1)$-ES. It has only one individual in population and on each iteration it creates new individual by mutating each bit of the current individual with probability of $\frac{\lambda}{n}$, where $\lambda$ is a mutation rate that is a fixed constant during an algorithm run. It accepts the new individual if and only if its fitess value is not worse then the fitnes value of the curent individual.

At first we considered \XdK function. But it seemed too complicated, so we simplified it to the following function $F$ that has some parameter $k$, as \XdK does:

\begin{align*}
  F(x) =
  \begin{cases}
    \OM(x), \text{ if } \OM(x) \le n - k \\
    n - k, \text{ if } n - k < \OM(x) < n \\
    n, \text{ if } \OM(x) = n \\
  \end{cases}
\end{align*}

Our aim is to find the optimal value of the mutation rate $\lambda$ that minimizes the expected runtime of the algorithm. If we reach our goal, we will be able to compare it the Fast Genetic Algorithm.

\section{First subproblem}
As passing over the plateau in the end of optimization process requires most time (at least $\Omega(n^2)$ for $k \ge 2$, while reaching the plateau requires only $O(n \log n)$), we are most interested in finding optimal $\lambda$ for passing the plateau. Attempts to find it that have been made by this moment are described in this document

\section{Notation}

We consider a markov chain that illustrates the behaviour of the considered algorithm on the plateau of the function $F$. It has $k + 1$ states that are numbered from $0$ to $k - 1$. If the algorithm is in state $i$, then the current individual has exactly $n - k + i$ one-bits in it.

By $p_i^j$ we imply the probability that algorithm that is in state $i$ will get to state $j$ on the next iteration.

The exact value of this probabilities is quite complicated:

\begin{align*}
  p_i^j = \begin{cases}
    \sum\limits_{m = 0}^{k - j} \binom{k - i}{j - i + m} \binom{n - k + i}{m} \left(\frac{\lambda}{n}\right)^{j - i + 2m} \left(1 - \frac{\lambda}{n}\right)^{n - j + i - 2m}, \text{ if } j > i \\
      \sum\limits_{m = 0}^{k - i} \binom{k - i}{m} \binom{n - k + i}{i - j + m} \left(\frac{\lambda}{n}\right)^{i - j + 2m} \left(1 - \frac{\lambda}{n}\right)^{n - i + j - 2m}, \text{ if } j < i \\
      1 - \sum\limits_{m = 0, m \ne i}^{k + 1} p_i^m, \text{ if } j = i \\
  \end{cases}
\end{align*}
\section{Bruteforce}



\end{document}
