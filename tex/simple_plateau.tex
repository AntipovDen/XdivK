\documentclass{article}

\usepackage{amsmath}
\usepackage{amsthm}
\usepackage{algorithmicx}
\usepackage{algpseudocode}
\usepackage{amsfonts}

\usepackage{flushend}
\usepackage{graphicx}
\usepackage{graphics}
\usepackage{url}
\usepackage[table]{xcolor}
\usepackage{xspace}
\usepackage[T2A]{fontenc}

\usepackage{algorithm}
\usepackage[english]{babel}

\usepackage[utf8]{inputenc}
\usepackage[backend=bibtex]{biblatex}

\usepackage{tabularx}

\newcommand{\OM}{\textsc{OneMax}\xspace}
 \newcommand{\J}{\textsc{Jump}\xspace}
\newcommand{\LB}{\textsc{LeftBridge}\xspace}
\newcommand{\RB}{\textsc{RightBridge}\xspace}
\newcommand{\EARL}{\textsc{EA+RL}\xspace}
\newcommand{\RLS}{\textsc{RLS}\xspace}
\newcommand{\OMZM}{\textsc{OneMax+ZeroMax}\xspace}
\newcommand{\XdK}{\textsc{XdivK}\xspace}

\newtheorem{theorem}{Theorem}
\newtheorem{lemma}{Lemma}
\newtheorem{definition}{Definition}
\newtheorem{corollary}{Corollary}
\addbibresource{bibliography.bib}
\allowdisplaybreaks

\begin{document}

\section{Drift Analysis on a Simple Plateau}
Consider the following problem. We have some integer number $x$ that is initially zero. Then we iterativly either increase it by one or decrease it by one randomly with equal probabilities. That means that if we have number $x_t$ on iteration $t$, then we have $\Pr[x_{t + 1} = x_t + 1] = \Pr[x_{t + 1} = x_t - 1] = \frac{1}{2}.$

The question is how long will it take $x$ to become grater than some number $n$ by its absolute value.

To answer this question consider all the integer numbers from $[-(n + 1)..(n + 1)]$ as states of some Markov chain. Transition probabilities are folowing:

\begin{align*}
  \Pr[i \to j] =
  \begin{cases}
    \frac{1}{2}, \text{ if } |i - j| = 1 \text{ and } |i| \ne n + 1, \\
    0, \text{ otherwise.}\\
  \end{cases}
\end{align*}

So we wonder, how much time will it take to reach either state $(n + 1)$ or state $-(n + 1)$ starting from state $0.$ To find it out let us use additive drift theorem. Let us define the following potential function for every state $i$ of the markov chain:

\begin{align*}
  \Phi(i) =
  \begin{cases}
    \sum\limits_{j = |i|}^n (2j + 1), \text{ if } |i| \ne n + 1, \\
    0, \text{ othewise.}\\
  \end{cases}
\end{align*}

Notice that $\Phi(|i|) - \Phi(|i| + 1) = 2|i| + 1$ for all $i \in [-n..n].$
The expected difference of the potential function, if $x = i \ne 0,$ is

\begin{align*}
  \delta = E(\Delta\Phi) = \frac{1}{2} (2|i| + 1) - \frac{1}{2}(2(|i| - 1) + 1) = 1.
\end{align*}

If $i = 0:$


\begin{align*}
  \delta = E(\Delta\Phi) = \frac{1}{2} 1 + \frac{1}{2} 1 = 1.
\end{align*}

Also
\begin{align*}
  \max \Phi(i) = \Phi(0) = \sum\limits_{j = 0}^n (2j + 1) = (n + 1) + 2 \frac{n}{2} (n + 1) = (n + 1)^2.
\end{align*}

It gives us the exact expected runtime $T = \frac{\max\Phi(i)}{\delta} = (n + 1)^2.$

\section{Applying the Plateau Theorem}

If we try to apply a plateau theorem, then we should consider states from $[-n..n]$ as a plateau and getting to state $(n + 1)$ or state $-(n + 1)$ as an escape from the plateau. In this case we discover that the stationary conditional disrtibution over the states is following:

\begin{align*}
  p_i^* =
  \begin{cases}
    \frac{1}{n - 1}, \text{ if } i \ne n, \\
    \frac{1}{2(n - 1)}, \text{ if } i = n. \\
  \end{cases}
\end{align*}

Let us check it only for non-negative $i$ to simplify notation.
\begin{itemize}
  \item If $i < (n - 1),$ then $p_i(t + 1) = \frac{1}{2} p_{i - 1}(t) + \frac{1}{2} p_{i + 1}(t) = \frac{1}{2}\frac{1}{n - 1} + \frac{1}{2}\frac{1}{n - 1} = p_i(t).$
  \item If $i = (n - 1),$ then $p_i(t + 1) = \frac{1}{2} p_{i - 1}(t) + 1 p_{i + 1}(t) = \frac{1}{2}\frac{1}{n - 1} + \frac{1}{2(n - 1)} = p_i(t).$
  \item If $i = n,$ then $p_i(t + 1) = \frac{1}{2} p_{i - 1}(t)= \frac{1}{2}\frac{1}{n - 1} = p_i(t).$
\end{itemize}

Now let us define $p_i^{n + 1}$ as the probability to escape the plateau from state~$i$. $p_n^{n + 1} = \frac{1}{2}$ and for any other state it is zero. Summing up all that we said above gives us the following result: if initial distribution is the stationary one, then the expected number of iterations before escaping the plateau is

\begin{align*}
  T = \frac{\sum\limits_{i = -n}^n (1 - p_i^{n + 1})}{\sum\limits_{i = -n}^n p_i^{n + 1}(1 - p_i^{n + 1})} = \frac{2n + 1}{1/2} = 4n + 2.
\end{align*}

We have got linear runtime that is totally contradicts the result shown into the previous section. What is the nature of this contradiction?

Using the plateau theorem we assumed that we have at least $\frac{1}{2(n - 1)}$ probability to be in one of the border states: $n$ or $-n,$ and later exactly this probability played role in the total expected runtime. But how much iterations do we need so that probability to be in the border states is $\Omega(1/n)?$

To answer this question let us notice that $x$ is a martingale, as long as $E(x_{t + 1}|x_t) = \frac{1}{2}(x_t + 1) + \frac{1}{2}(x_t - 1) = x_t.$

So we can use Azuma–Hoeffding inequality to bound the probability that after $k$ iterations $|x_k| = |x_k - x_0| \ge n:$

\begin{align*}
  \Pr[|x_k| \ge n] \le 2e^{-\frac{n^2}{2k}}.
\end{align*}

If we have $k = o\left(\frac{n^2}{\ln(n)}\right),$ then this probability is:

\begin{align*}
  \Pr[|x_k| \ge n] \le 2e^{-\frac{n^2}{o\left(\frac{n^2}{\ln n}\right)}} = 2e^{-\omega(\ln(n))} = o(1/n).
\end{align*}

\textbf{NB:} \textit{equalities $\frac{n^2}{o\left(\frac{n^2}{\ln n}\right)} = \omega(\ln(n))$ and $2e^{-\omega(\ln(n))} = o(1/n)$ can be justified by definitions of $o()$ and $\omega()$.}

So we need at least $\Omega\left(\frac{n^2}{\log(n)}\right)$ iterations to reach the probability to escape the plateau at least $\Omega(1/n).$ And that is why we can not apply our plateau theorem to this toy problem.
\end{document}
