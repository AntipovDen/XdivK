\documentclass{article}

\usepackage{amsmath}
\usepackage{amsthm}
\usepackage{algorithmicx}
\usepackage{algpseudocode}

\usepackage{flushend}
\usepackage{graphicx}
\usepackage{graphics}
\usepackage{url}
\usepackage[table]{xcolor}
\usepackage{xspace}
\usepackage[T2A]{fontenc}

\usepackage{algorithm}
\usepackage[english]{babel}

\usepackage[utf8]{inputenc}
\usepackage[backend=bibtex]{biblatex}

\newcommand{\OM}{\textsc{OneMax}\xspace}
 \newcommand{\J}{\textsc{Jump}\xspace}
\newcommand{\LB}{\textsc{LeftBridge}\xspace}
\newcommand{\RB}{\textsc{RightBridge}\xspace}
\newcommand{\EARL}{\textsc{EA+RL}\xspace}
\newcommand{\RLS}{\textsc{RLS}\xspace}
\newcommand{\OMZM}{\textsc{OneMax+ZeroMax}\xspace}
\newcommand{\XdK}{\textsc{XdivK}\xspace}

\newtheorem{theorem}{Theorem}
\newtheorem{lemma}{Lemma}
\addbibresource{bibliography.bib}
\allowdisplaybreaks

\begin{document}

We say that distribution $\pi_1$ is dominated by distribution $\pi_2$ ($\pi_1 \preceq \pi_2$), if for every constant $a$ and for two random values $X_1$ and $X_2$ that are generated by $\pi_1$ and $\pi_2$ respectively it is true that:
$$\Pr[X_1 \ge a] \le \Pr[X_2 \ge a],$$
which is similar to
$$\Pr[X_1 \le a] \ge \Pr[X_2 \le a].$$

Consider the following setting. Let the length of individual $n$ be large enough, for example, $n = 10^9.$ Let $k = 100$ (just not too small value) and $\lambda = 50$ (it is also just not too small value).

Consder the following distributions over the set of $k + 1$ states. Distribution $\pi_1$: the probability to be in state $i = 24$ ($i$ jsut has to be greater than $\lambda$) equals to one and probability to be in any other state is zero. Distribution $\pi_2$: the probability to be in state $i + 1= 25$ equals to one and probability to be in any other state is zero. It is easy to see that $\pi_1 \preceq \pi_2$, as for $a \in [i, i + 1)$ it is clear that $1 = \Pr[\pi_1 \le a] > \Pr[\pi_2 \le a] = 0,$ and for any other $a$ the probabilities are equal.

However, if we perform one iteration, then we get two new distributions $\pi_1 P = (p_i^0, p_i^1, \dots, p_i^k)$ and $\pi_2 P = (p_{i + 1}^0, p_{i + 1}^1, \dots, p_{i + 1}^k)$.
Now notice that
$$p_{i + 1}^0 \approx \binom{n - k + i + 1}{i + 1}\left(\frac{\lambda}{n}\right)^{i + 1} = \binom{n - k + i}{i} \left(\frac{\lambda}{n}\right)^i \cdot \frac{(n - k + i + 1)\lambda}{(i + 1)n} \approx \frac{(n - k + i)\lambda}{(i + 1)n} \cdot p_i^0.$$

In our setting $\frac{(n - k + i)\lambda}{(i + 1)n} \approx 2 > 1$, so $p_{i + 1}^0 > p_i^0.$ Thus, for $a = 0$ we have $\Pr[\pi_1 P \le 0] < \Pr[\pi_2 P \le 0]$ that means that $\pi_2 P$ does not dominate $\pi_1 P.$

Actually we have discovered a strange fact that it is easier to fall back to state $0$ from the states with greater number as soon as the number is not greater than $\lambda$, than from the states that are closer to state $0$

\end{document}
